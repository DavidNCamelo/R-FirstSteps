% Options for packages loaded elsewhere
% Options for packages loaded elsewhere
\PassOptionsToPackage{unicode}{hyperref}
\PassOptionsToPackage{hyphens}{url}
\PassOptionsToPackage{dvipsnames,svgnames,x11names}{xcolor}
%
\documentclass[
  letterpaper,
  DIV=11,
  numbers=noendperiod]{scrartcl}
\usepackage{xcolor}
\usepackage{amsmath,amssymb}
\setcounter{secnumdepth}{-\maxdimen} % remove section numbering
\usepackage{iftex}
\ifPDFTeX
  \usepackage[T1]{fontenc}
  \usepackage[utf8]{inputenc}
  \usepackage{textcomp} % provide euro and other symbols
\else % if luatex or xetex
  \usepackage{unicode-math} % this also loads fontspec
  \defaultfontfeatures{Scale=MatchLowercase}
  \defaultfontfeatures[\rmfamily]{Ligatures=TeX,Scale=1}
\fi
\usepackage{lmodern}
\ifPDFTeX\else
  % xetex/luatex font selection
\fi
% Use upquote if available, for straight quotes in verbatim environments
\IfFileExists{upquote.sty}{\usepackage{upquote}}{}
\IfFileExists{microtype.sty}{% use microtype if available
  \usepackage[]{microtype}
  \UseMicrotypeSet[protrusion]{basicmath} % disable protrusion for tt fonts
}{}
\makeatletter
\@ifundefined{KOMAClassName}{% if non-KOMA class
  \IfFileExists{parskip.sty}{%
    \usepackage{parskip}
  }{% else
    \setlength{\parindent}{0pt}
    \setlength{\parskip}{6pt plus 2pt minus 1pt}}
}{% if KOMA class
  \KOMAoptions{parskip=half}}
\makeatother
% Make \paragraph and \subparagraph free-standing
\makeatletter
\ifx\paragraph\undefined\else
  \let\oldparagraph\paragraph
  \renewcommand{\paragraph}{
    \@ifstar
      \xxxParagraphStar
      \xxxParagraphNoStar
  }
  \newcommand{\xxxParagraphStar}[1]{\oldparagraph*{#1}\mbox{}}
  \newcommand{\xxxParagraphNoStar}[1]{\oldparagraph{#1}\mbox{}}
\fi
\ifx\subparagraph\undefined\else
  \let\oldsubparagraph\subparagraph
  \renewcommand{\subparagraph}{
    \@ifstar
      \xxxSubParagraphStar
      \xxxSubParagraphNoStar
  }
  \newcommand{\xxxSubParagraphStar}[1]{\oldsubparagraph*{#1}\mbox{}}
  \newcommand{\xxxSubParagraphNoStar}[1]{\oldsubparagraph{#1}\mbox{}}
\fi
\makeatother

\usepackage{color}
\usepackage{fancyvrb}
\newcommand{\VerbBar}{|}
\newcommand{\VERB}{\Verb[commandchars=\\\{\}]}
\DefineVerbatimEnvironment{Highlighting}{Verbatim}{commandchars=\\\{\}}
% Add ',fontsize=\small' for more characters per line
\usepackage{framed}
\definecolor{shadecolor}{RGB}{241,243,245}
\newenvironment{Shaded}{\begin{snugshade}}{\end{snugshade}}
\newcommand{\AlertTok}[1]{\textcolor[rgb]{0.68,0.00,0.00}{#1}}
\newcommand{\AnnotationTok}[1]{\textcolor[rgb]{0.37,0.37,0.37}{#1}}
\newcommand{\AttributeTok}[1]{\textcolor[rgb]{0.40,0.45,0.13}{#1}}
\newcommand{\BaseNTok}[1]{\textcolor[rgb]{0.68,0.00,0.00}{#1}}
\newcommand{\BuiltInTok}[1]{\textcolor[rgb]{0.00,0.23,0.31}{#1}}
\newcommand{\CharTok}[1]{\textcolor[rgb]{0.13,0.47,0.30}{#1}}
\newcommand{\CommentTok}[1]{\textcolor[rgb]{0.37,0.37,0.37}{#1}}
\newcommand{\CommentVarTok}[1]{\textcolor[rgb]{0.37,0.37,0.37}{\textit{#1}}}
\newcommand{\ConstantTok}[1]{\textcolor[rgb]{0.56,0.35,0.01}{#1}}
\newcommand{\ControlFlowTok}[1]{\textcolor[rgb]{0.00,0.23,0.31}{\textbf{#1}}}
\newcommand{\DataTypeTok}[1]{\textcolor[rgb]{0.68,0.00,0.00}{#1}}
\newcommand{\DecValTok}[1]{\textcolor[rgb]{0.68,0.00,0.00}{#1}}
\newcommand{\DocumentationTok}[1]{\textcolor[rgb]{0.37,0.37,0.37}{\textit{#1}}}
\newcommand{\ErrorTok}[1]{\textcolor[rgb]{0.68,0.00,0.00}{#1}}
\newcommand{\ExtensionTok}[1]{\textcolor[rgb]{0.00,0.23,0.31}{#1}}
\newcommand{\FloatTok}[1]{\textcolor[rgb]{0.68,0.00,0.00}{#1}}
\newcommand{\FunctionTok}[1]{\textcolor[rgb]{0.28,0.35,0.67}{#1}}
\newcommand{\ImportTok}[1]{\textcolor[rgb]{0.00,0.46,0.62}{#1}}
\newcommand{\InformationTok}[1]{\textcolor[rgb]{0.37,0.37,0.37}{#1}}
\newcommand{\KeywordTok}[1]{\textcolor[rgb]{0.00,0.23,0.31}{\textbf{#1}}}
\newcommand{\NormalTok}[1]{\textcolor[rgb]{0.00,0.23,0.31}{#1}}
\newcommand{\OperatorTok}[1]{\textcolor[rgb]{0.37,0.37,0.37}{#1}}
\newcommand{\OtherTok}[1]{\textcolor[rgb]{0.00,0.23,0.31}{#1}}
\newcommand{\PreprocessorTok}[1]{\textcolor[rgb]{0.68,0.00,0.00}{#1}}
\newcommand{\RegionMarkerTok}[1]{\textcolor[rgb]{0.00,0.23,0.31}{#1}}
\newcommand{\SpecialCharTok}[1]{\textcolor[rgb]{0.37,0.37,0.37}{#1}}
\newcommand{\SpecialStringTok}[1]{\textcolor[rgb]{0.13,0.47,0.30}{#1}}
\newcommand{\StringTok}[1]{\textcolor[rgb]{0.13,0.47,0.30}{#1}}
\newcommand{\VariableTok}[1]{\textcolor[rgb]{0.07,0.07,0.07}{#1}}
\newcommand{\VerbatimStringTok}[1]{\textcolor[rgb]{0.13,0.47,0.30}{#1}}
\newcommand{\WarningTok}[1]{\textcolor[rgb]{0.37,0.37,0.37}{\textit{#1}}}

\usepackage{longtable,booktabs,array}
\usepackage{calc} % for calculating minipage widths
% Correct order of tables after \paragraph or \subparagraph
\usepackage{etoolbox}
\makeatletter
\patchcmd\longtable{\par}{\if@noskipsec\mbox{}\fi\par}{}{}
\makeatother
% Allow footnotes in longtable head/foot
\IfFileExists{footnotehyper.sty}{\usepackage{footnotehyper}}{\usepackage{footnote}}
\makesavenoteenv{longtable}
\usepackage{graphicx}
\makeatletter
\newsavebox\pandoc@box
\newcommand*\pandocbounded[1]{% scales image to fit in text height/width
  \sbox\pandoc@box{#1}%
  \Gscale@div\@tempa{\textheight}{\dimexpr\ht\pandoc@box+\dp\pandoc@box\relax}%
  \Gscale@div\@tempb{\linewidth}{\wd\pandoc@box}%
  \ifdim\@tempb\p@<\@tempa\p@\let\@tempa\@tempb\fi% select the smaller of both
  \ifdim\@tempa\p@<\p@\scalebox{\@tempa}{\usebox\pandoc@box}%
  \else\usebox{\pandoc@box}%
  \fi%
}
% Set default figure placement to htbp
\def\fps@figure{htbp}
\makeatother





\setlength{\emergencystretch}{3em} % prevent overfull lines

\providecommand{\tightlist}{%
  \setlength{\itemsep}{0pt}\setlength{\parskip}{0pt}}



 


\KOMAoption{captions}{tableheading}
\makeatletter
\@ifpackageloaded{caption}{}{\usepackage{caption}}
\AtBeginDocument{%
\ifdefined\contentsname
  \renewcommand*\contentsname{Table of contents}
\else
  \newcommand\contentsname{Table of contents}
\fi
\ifdefined\listfigurename
  \renewcommand*\listfigurename{List of Figures}
\else
  \newcommand\listfigurename{List of Figures}
\fi
\ifdefined\listtablename
  \renewcommand*\listtablename{List of Tables}
\else
  \newcommand\listtablename{List of Tables}
\fi
\ifdefined\figurename
  \renewcommand*\figurename{Figure}
\else
  \newcommand\figurename{Figure}
\fi
\ifdefined\tablename
  \renewcommand*\tablename{Table}
\else
  \newcommand\tablename{Table}
\fi
}
\@ifpackageloaded{float}{}{\usepackage{float}}
\floatstyle{ruled}
\@ifundefined{c@chapter}{\newfloat{codelisting}{h}{lop}}{\newfloat{codelisting}{h}{lop}[chapter]}
\floatname{codelisting}{Listing}
\newcommand*\listoflistings{\listof{codelisting}{List of Listings}}
\makeatother
\makeatletter
\makeatother
\makeatletter
\@ifpackageloaded{caption}{}{\usepackage{caption}}
\@ifpackageloaded{subcaption}{}{\usepackage{subcaption}}
\makeatother
\usepackage{bookmark}
\IfFileExists{xurl.sty}{\usepackage{xurl}}{} % add URL line breaks if available
\urlstyle{same}
\hypersetup{
  pdftitle={Introducción y Conceptos Básicos R},
  colorlinks=true,
  linkcolor={blue},
  filecolor={Maroon},
  citecolor={Blue},
  urlcolor={Blue},
  pdfcreator={LaTeX via pandoc}}


\title{Introducción y Conceptos Básicos R}
\author{}
\date{}
\begin{document}
\maketitle


\section{Consideraciones de R}\label{consideraciones-de-r}

\subsection{Sintaxis básica}\label{sintaxis-buxe1sica}

Las variables en R funcionan como elemntos para almacenar valores, en lo
posible reutilizables

las funciones son elementos, también reutilizables, pero adicional
pueden establecerse con características propias en su interior para
realizar cualquier tipo de operación o proceso,

\begin{Shaded}
\begin{Highlighting}[]
\CommentTok{\#Números enteros}
\NormalTok{entero }\OtherTok{\textless{}{-}} \DecValTok{3}
\CommentTok{\# Validar tipo}
\FunctionTok{print}\NormalTok{(}\FunctionTok{paste0}\NormalTok{(}\StringTok{\textquotesingle{}Tipo de variable entero:\textquotesingle{}}\NormalTok{, }\FunctionTok{class}\NormalTok{(entero)))}
\end{Highlighting}
\end{Shaded}

\begin{verbatim}
[1] "Tipo de variable entero:numeric"
\end{verbatim}

\begin{Shaded}
\begin{Highlighting}[]
\CommentTok{\# Números flotantes (decimales=)}
\NormalTok{flotante }\OtherTok{\textless{}{-}} \FloatTok{3.6}
\FunctionTok{print}\NormalTok{(}\FunctionTok{paste0}\NormalTok{(}\StringTok{\textquotesingle{}Tipo de variable flotante:\textquotesingle{}}\NormalTok{, }\FunctionTok{class}\NormalTok{(flotante)))}
\end{Highlighting}
\end{Shaded}

\begin{verbatim}
[1] "Tipo de variable flotante:numeric"
\end{verbatim}

\begin{Shaded}
\begin{Highlighting}[]
\CommentTok{\# Texto}
\NormalTok{texto }\OtherTok{\textless{}{-}} \StringTok{\textquotesingle{}Hola mundo\textquotesingle{}}
\FunctionTok{print}\NormalTok{(}\FunctionTok{paste0}\NormalTok{(}\StringTok{\textquotesingle{}Tipo de variable texto:\textquotesingle{}}\NormalTok{, }\FunctionTok{class}\NormalTok{(texto)))}
\end{Highlighting}
\end{Shaded}

\begin{verbatim}
[1] "Tipo de variable texto:character"
\end{verbatim}

\begin{Shaded}
\begin{Highlighting}[]
\CommentTok{\# Lógicos}
\FunctionTok{print}\NormalTok{(}\DecValTok{7} \SpecialCharTok{==} \DecValTok{8}\NormalTok{)}
\end{Highlighting}
\end{Shaded}

\begin{verbatim}
[1] FALSE
\end{verbatim}

\begin{Shaded}
\begin{Highlighting}[]
\FunctionTok{print}\NormalTok{(}\DecValTok{8} \SpecialCharTok{==} \DecValTok{8}\NormalTok{)}
\end{Highlighting}
\end{Shaded}

\begin{verbatim}
[1] TRUE
\end{verbatim}

\begin{Shaded}
\begin{Highlighting}[]
\CommentTok{\#Crear vectores}
\NormalTok{mixto }\OtherTok{\textless{}{-}} \FunctionTok{c}\NormalTok{(}\DecValTok{2}\NormalTok{, }\DecValTok{4}\NormalTok{, }\StringTok{\textquotesingle{}manzanas\textquotesingle{}}\NormalTok{, }\StringTok{\textquotesingle{}bananas\textquotesingle{}}\NormalTok{)}
\FunctionTok{print}\NormalTok{(}\StringTok{\textquotesingle{}Contenido de mixto\textquotesingle{}}\NormalTok{)}
\end{Highlighting}
\end{Shaded}

\begin{verbatim}
[1] "Contenido de mixto"
\end{verbatim}

\begin{Shaded}
\begin{Highlighting}[]
\FunctionTok{print}\NormalTok{(mixto)}
\end{Highlighting}
\end{Shaded}

\begin{verbatim}
[1] "2"        "4"        "manzanas" "bananas" 
\end{verbatim}

\begin{Shaded}
\begin{Highlighting}[]
\FunctionTok{print}\NormalTok{(}\StringTok{\textquotesingle{}Segundo elemento del vector mixto:\textquotesingle{}}\NormalTok{)}
\end{Highlighting}
\end{Shaded}

\begin{verbatim}
[1] "Segundo elemento del vector mixto:"
\end{verbatim}

\begin{Shaded}
\begin{Highlighting}[]
\FunctionTok{print}\NormalTok{(mixto[}\DecValTok{2}\NormalTok{])}
\end{Highlighting}
\end{Shaded}

\begin{verbatim}
[1] "4"
\end{verbatim}

\begin{Shaded}
\begin{Highlighting}[]
\CommentTok{\# Creas matrices}
\NormalTok{matriz\_random }\OtherTok{\textless{}{-}} \FunctionTok{matrix}\NormalTok{( mixto, }\AttributeTok{nrow =} \DecValTok{2}\NormalTok{, }\AttributeTok{ncol =} \DecValTok{6}\NormalTok{)}
\FunctionTok{print}\NormalTok{(}\StringTok{\textquotesingle{}Contenido de mmatriz\_random\textquotesingle{}}\NormalTok{)}
\end{Highlighting}
\end{Shaded}

\begin{verbatim}
[1] "Contenido de mmatriz_random"
\end{verbatim}

\begin{Shaded}
\begin{Highlighting}[]
\FunctionTok{print}\NormalTok{(matriz\_random)}
\end{Highlighting}
\end{Shaded}

\begin{verbatim}
     [,1] [,2]       [,3] [,4]       [,5] [,6]      
[1,] "2"  "manzanas" "2"  "manzanas" "2"  "manzanas"
[2,] "4"  "bananas"  "4"  "bananas"  "4"  "bananas" 
\end{verbatim}

\begin{Shaded}
\begin{Highlighting}[]
\CommentTok{\# Funciones + for + condicionales}
\NormalTok{valores }\OtherTok{\textless{}{-}} \FunctionTok{c}\NormalTok{(}\DecValTok{20}\NormalTok{, }\DecValTok{18}\NormalTok{, }\DecValTok{25}\NormalTok{, }\DecValTok{40}\NormalTok{, }\DecValTok{49}\NormalTok{, }\DecValTok{68}\NormalTok{)}

\CommentTok{\# Una función se crea a través de = y asignación de parámetros}
\NormalTok{es\_par }\OtherTok{\textless{}{-}} \ControlFlowTok{function}\NormalTok{(valores) \{}
    \CommentTok{\# Ciclo for }
    \ControlFlowTok{for}\NormalTok{ (i }\ControlFlowTok{in} \DecValTok{1}\SpecialCharTok{:}\FunctionTok{length}\NormalTok{(valores)) \{}
        \CommentTok{\# Condicional}
        \ControlFlowTok{if}\NormalTok{(valores[i]}\SpecialCharTok{\%\%}\DecValTok{2} \SpecialCharTok{==} \DecValTok{0}\NormalTok{)\{}
            \FunctionTok{print}\NormalTok{(}\FunctionTok{paste0}\NormalTok{(}\StringTok{\textquotesingle{}El valor \textquotesingle{}}\NormalTok{, valores[i], }\StringTok{\textquotesingle{} es par\textquotesingle{}}\NormalTok{))}
\NormalTok{        \} }\ControlFlowTok{else}\NormalTok{ \{}
            \FunctionTok{print}\NormalTok{(}\FunctionTok{paste0}\NormalTok{(}\StringTok{\textquotesingle{}El valor \textquotesingle{}}\NormalTok{, valores[i], }\StringTok{\textquotesingle{} es inparpar\textquotesingle{}}\NormalTok{))}
\NormalTok{        \}}
\NormalTok{    \}}
\NormalTok{\}}

\CommentTok{\# Llamar y usar función}
\FunctionTok{es\_par}\NormalTok{(valores)}
\end{Highlighting}
\end{Shaded}

\begin{verbatim}
[1] "El valor 20 es par"
[1] "El valor 18 es par"
[1] "El valor 25 es inparpar"
[1] "El valor 40 es par"
[1] "El valor 49 es inparpar"
[1] "El valor 68 es par"
\end{verbatim}

\subsection{Base vs tidyverse}\label{base-vs-tidyverse}

En R, existen dos enfoques principales para la manipulación y análisis
de datos: Base R y el Tidyverse. Ambos son poderosos, pero difieren
significativamente en su sintaxis, filosofía y estilo de programación.
Comprender estas diferencias es clave para escribir código R eficiente y
legible.

\subsubsection{¿Qué es Base R?}\label{quuxe9-es-base-r}

Base R se refiere a las funciones y paquetes que vienen preinstalados
con R. Es el conjunto fundamental de herramientas que ha existido desde
los inicios de R. Sus funciones a menudo operan directamente sobre
vectores, matrices y data frames utilizando corchetes ({[}{]}) y el
operador \$ para acceder a elementos o columnas.

\subsubsection{¿Qué es Tidyverse?}\label{quuxe9-es-tidyverse}

El Tidyverse es una colección de paquetes de R (como dplyr, ggplot2,
tidyr, readr, entre otros) que comparten una filosofía de diseño común:
hacer que la manipulación de datos sea más intuitiva, consistente y
legible. Se enfoca en el concepto de ``datos ordenados'' (tidy data) y
utiliza el operador
\texttt{\%\textgreater{}\%\ (pipe)\ o\ \textbar{}\textgreater{}\ (en\ R\ 4.1+)}
para encadenar múltiples operaciones de forma secuencial.

\subsubsection{Ejemplos Prácticos}\label{ejemplos-pruxe1cticos}

Para ilustrar las diferencias, usaremos un pequeño data.frame de
ejemplo:

\begin{Shaded}
\begin{Highlighting}[]
\CommentTok{\# Crear un data frame de ejemplo}
\NormalTok{datos }\OtherTok{\textless{}{-}} \FunctionTok{data.frame}\NormalTok{(}
  \AttributeTok{ID =} \DecValTok{1}\SpecialCharTok{:}\DecValTok{5}\NormalTok{,}
  \AttributeTok{Producto =} \FunctionTok{c}\NormalTok{(}\StringTok{"Manzana"}\NormalTok{, }\StringTok{"Banana"}\NormalTok{, }\StringTok{"Naranja"}\NormalTok{, }\StringTok{"Manzana"}\NormalTok{, }\StringTok{"Kiwi"}\NormalTok{),}
  \AttributeTok{Precio =} \FunctionTok{c}\NormalTok{(}\FloatTok{1.5}\NormalTok{, }\FloatTok{0.75}\NormalTok{, }\FloatTok{1.2}\NormalTok{, }\FloatTok{1.5}\NormalTok{, }\FloatTok{2.0}\NormalTok{),}
  \AttributeTok{Cantidad =} \FunctionTok{c}\NormalTok{(}\DecValTok{10}\NormalTok{, }\DecValTok{20}\NormalTok{, }\DecValTok{15}\NormalTok{, }\DecValTok{5}\NormalTok{, }\DecValTok{8}\NormalTok{)}
\NormalTok{)}
\FunctionTok{print}\NormalTok{(datos)}
\end{Highlighting}
\end{Shaded}

\begin{verbatim}
  ID Producto Precio Cantidad
1  1  Manzana   1.50       10
2  2   Banana   0.75       20
3  3  Naranja   1.20       15
4  4  Manzana   1.50        5
5  5     Kiwi   2.00        8
\end{verbatim}

\subsection{1. Filtrar Filas y Seleccionar
Columnas}\label{filtrar-filas-y-seleccionar-columnas}

\textbf{Objetivo:} Obtener el ID y Producto de los ítems con Cantidad
mayor a 10.

\paragraph{Con Base R:}\label{con-base-r}

\begin{Shaded}
\begin{Highlighting}[]
\CommentTok{\# Filtrar filas y seleccionar columnas en Base R}
\NormalTok{resultados\_baseR }\OtherTok{\textless{}{-}}\NormalTok{ datos[datos}\SpecialCharTok{$}\NormalTok{Cantidad }\SpecialCharTok{\textgreater{}} \DecValTok{10}\NormalTok{, }\FunctionTok{c}\NormalTok{(}\StringTok{"ID"}\NormalTok{, }\StringTok{"Producto"}\NormalTok{)]}
\FunctionTok{print}\NormalTok{(resultados\_baseR)}
\end{Highlighting}
\end{Shaded}

\begin{verbatim}
  ID Producto
2  2   Banana
3  3  Naranja
\end{verbatim}

\paragraph{Con Tidyverse (dplyr):}\label{con-tidyverse-dplyr}

\begin{Shaded}
\begin{Highlighting}[]
\CommentTok{\# Cargar la librería dplyr}
\FunctionTok{library}\NormalTok{(dplyr)}

\CommentTok{\# Filtrar filas y seleccionar columnas con Tidyverse (dplyr)}
\NormalTok{resultados\_tidyverse }\OtherTok{\textless{}{-}}\NormalTok{ datos }\SpecialCharTok{\%\textgreater{}\%}
  \FunctionTok{filter}\NormalTok{(Cantidad }\SpecialCharTok{\textgreater{}} \DecValTok{10}\NormalTok{) }\SpecialCharTok{\%\textgreater{}\%}
  \FunctionTok{select}\NormalTok{(ID, Producto)}
\FunctionTok{print}\NormalTok{(resultados\_tidyverse)}
\end{Highlighting}
\end{Shaded}

\begin{verbatim}
  ID Producto
1  2   Banana
2  3  Naranja
\end{verbatim}

\subsection{2. Agrupar y Resumir Datos}\label{agrupar-y-resumir-datos}

\textbf{Objetivo:} Calcular el Precio total por Producto.

\paragraph{Con Base R:}\label{con-base-r-1}

\begin{Shaded}
\begin{Highlighting}[]
\CommentTok{\# Agrupar y resumir en Base R}
\NormalTok{precios\_totales\_baseR }\OtherTok{\textless{}{-}} \FunctionTok{aggregate}\NormalTok{(Precio }\SpecialCharTok{\textasciitilde{}}\NormalTok{ Producto, }\AttributeTok{data =}\NormalTok{ datos, }\AttributeTok{FUN =}\NormalTok{ sum)}
\FunctionTok{print}\NormalTok{(precios\_totales\_baseR)}
\end{Highlighting}
\end{Shaded}

\begin{verbatim}
  Producto Precio
1   Banana   0.75
2     Kiwi   2.00
3  Manzana   3.00
4  Naranja   1.20
\end{verbatim}

\textbf{Explicación:} group\_by() define los grupos para la operación.
summarise() luego aplica la función de agregación (sum) a la columna
Precio dentro de cada grupo y crea una nueva columna PrecioTotal.
Nuevamente, el flujo es claro y secuencial.

\subsubsection{Comparación: Pros y
Contras}\label{comparaciuxf3n-pros-y-contras}

\begin{longtable}[]{@{}
  >{\raggedright\arraybackslash}p{(\linewidth - 4\tabcolsep) * \real{0.1754}}
  >{\raggedright\arraybackslash}p{(\linewidth - 4\tabcolsep) * \real{0.4123}}
  >{\raggedright\arraybackslash}p{(\linewidth - 4\tabcolsep) * \real{0.4123}}@{}}
\toprule\noalign{}
\begin{minipage}[b]{\linewidth}\raggedright
Característica
\end{minipage} & \begin{minipage}[b]{\linewidth}\raggedright
Base R
\end{minipage} & \begin{minipage}[b]{\linewidth}\raggedright
Tidyverse
\end{minipage} \\
\midrule\noalign{}
\endhead
\bottomrule\noalign{}
\endlastfoot
\textbf{Sintaxis} & Flexible, a menudo concisa pero puede ser menos
intuitiva para cadenas complejas. Usa \texttt{{[}{]}}, \texttt{\$}. &
Consistente, verbos claros (\texttt{filter}, \texttt{select},
\texttt{mutate}). Usa \texttt{\%\textgreater{}\%} o
\texttt{\textbar{}\textgreater{}}. Muy legible. \\
\textbf{Filosofía} & Operaciones directas sobre estructuras de datos. &
Enfoque en ``tidy data'' (datos ordenados), facilitando la manipulación
secuencial. \\
\textbf{Curva de Aprendizaje} & Puede ser empinada inicialmente para
manipulación de datos, pero fundamental para entender R. & Generalmente
más suave para tareas de manipulación de datos debido a su
consistencia. \\
\textbf{Legibilidad} & Puede volverse difícil de leer con muchas
anidaciones o llamadas a funciones. & Altamente legible debido al
encadenamiento de operaciones con el pipe. \\
\textbf{Dependencias} & No requiere instalar paquetes adicionales. &
Requiere instalar y cargar paquetes específicos (ej. \texttt{dplyr}). \\
\textbf{Rendimiento} & A menudo muy eficiente para operaciones
vectorizadas. & Generalmente muy bueno, con implementaciones optimizadas
para muchas funciones comunes. \\
\textbf{Flexibilidad} & Muy alta, ya que permite un control granular
sobre todos los aspectos. & Alta, pero guiada por una gramática de datos
específica. \\
\end{longtable}

\paragraph{¿Cuándo usar cada uno?}\label{cuuxe1ndo-usar-cada-uno}

\textbf{Usa Base R cuando:}

\begin{itemize}
\tightlist
\item
  Estás realizando operaciones muy simples y directas (ej. calcular la
  media de un vector).
\item
  Quieres un control muy granular y entender profundamente cómo R maneja
  los objetos y las operaciones.
\item
  Estás trabajando en entornos donde la instalación de paquetes
  adicionales es restringida o indeseable.
\item
  Necesitas escribir código para funciones muy específicas de R que aún
  no tienen un ``equivalente tidy'' directo o eficiente.
\end{itemize}

\textbf{Usa Tidyverse cuando:}

\begin{itemize}
\tightlist
\item
  Estás realizando tareas de manipulación de datos complejas o que
  involucran múltiples pasos.
\item
  La legibilidad y la reproducibilidad del código son prioritarias.
\item
  Estás trabajando con otros analistas de datos o científicos de datos
  que usan el Tidyverse (lo cual es muy común).
\item
  Necesitas hacer visualizaciones de datos avanzadas (ggplot2).
\item
  Buscas un flujo de trabajo consistente y eficiente para la limpieza,
  transformación y análisis de datos.
\end{itemize}

\textbf{En resumen:} Para la mayoría de las tareas de ciencia de datos y
análisis de datos en R, el Tidyverse se ha convertido en el estándar de
facto debido a su legibilidad, consistencia y eficiencia para la
manipulación de datos. Sin embargo, comprender los fundamentos de Base R
es esencial para cualquier usuario serio de R, ya que muchas funciones
del Tidyverse se construyen sobre o interactúan con conceptos de Base R.
La mejor práctica a menudo implica el uso de ambos, aprovechando las
fortalezas de cada uno según la necesidad.




\end{document}
